\newpage
\mychapter{Обсуждение результатов}

Результаты, полученные при малых частотах Раби (\ref{fig:350_30}), (\ref{fig:850_20}), как и ожидалось,
соответствуют результатам, полученным аналитически в двухуровневом
приближении (\ref{fig:two_level}). Паразитная заселенность не
используемого уровня практически не отличима от нуля. Форма графиков
для уровней $m_s$=$0$, $m_s$=$-1$ повторяет форму для тех же уровней в
двухуровневом приближении. Таким образом можно заключить, что расчеты, произведенные
в приближении вращающейся волны, применимы для экспериментов на малых
частотах Раби.

Однако и при частотах Раби, близких к частоте резонансного перехода, но
по прежнему малых по сравнению с энергией расщепления, паразитная
заселенность остается достаточно малой (\ref{fig:350_100}), (\ref{fig:850_150}), чтобы с высокой вероятностью
производить операцию NOT. Причем при таких значениях частоты Раби
скорость этой операции значительно выше, чем при низких значениях. С
другой стороны, нельзя бесконечно ускорять операцию за счет увеличения
частоты Раби, так как заселенность неиспользуемого уровня быстро
растет с ростом частоты Раби (\ref{fig:350_300}),
(\ref{fig:850_680}). В данной работе за оптимальную частоту была
выбрана частота Раби, соответствующая амплитуде микроволнового поля
170 Гс (\ref{fig:850_170}).

Была осуществлена попытка увеличить скорость операции NOT, варьируя
форму амплитуды микроволнового поля. С помощью прямоугольных импульсов
удалось достичь длительности операции 14 нс. Пилообразный сигнал
явился более удачным: с его помощью NOT был произведен за 12
нс. Наибольшее ускорение удалось получить при использовании гауссовой
формы амплитуды. Операция NOT заняла всего 6 нс.

В работе [\ref{lit:gigahertz}] были достигнуты похожие
результаты. Кроме того, авторы подтвердили свои результаты
экспериментальными данными. NV-центр был помещен в постоянное
магнитное поле 850 Гс, а переменное поле на резонансной частоте
имело амплитуду от 29 ГГц до 440 ГГц. В первом случае наблюдались
плавные синусоидальные осцилляции Раби, как на графике
(\ref{fig:850_20}). Этот результат соответствует теории в приближении
вращающейся волны. Во случае очень сильного переменного поля (440 ГГц)
виден сильный ангармонизм в колебаниях Раби. Например, результаты
работы [\ref{lit:gigahertz}] при переменном поле 223 ГГц сильно
напоминают график (\ref{fig:850_150}).

Кроме того, авторами работы [\ref{lit:gigahertz}] было произведено
исследование при прямоугольной и гауссовой форме микроволнового
импульса. Результатом явилось сверхбыстрое (<1нс) вращение спина при
гауссовой форме импульса. В данной работе субнаносекундных спиновых
вращений обнаружить не удалось, но общие результаты соответствуют
результатам работы [\ref{lit:gigahertz}].
