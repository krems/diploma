\newpage
\chapter*{Заключение}
\addcontentsline{toc}{chapter}{Заключение}
Общепринятой практикой при расчете двухуровневых систем является
сведение более сложных систем к системам с псевдоспином $\frac{1}{2}$
в приближении вращающейся волны. В этом приближении можно пренебречь
наличием других уровней у системы, выделив два. Подобным методом в
данной работе был произведен расчет заселенностей уровней NV-центра в
аналитической части работы. В результате такого сведения был получен
рисунок (\ref{fig:two_level}). Численное моделирование трехуровневой
системы в условиях, удовлетворяющих приближению вращающейся волны,
дало очень похожие результаты, что можно наблюдать на рисунке
(\ref{fig:850_20}).

Проведенное исследование показывает, что даже при относительно больших
(порядка частоты перехода) частотах Раби остается высокая вероятность
выполнения операции NOT. Паразитная заселенность уровней пренебрежимо
мала по сравнению с ``полезными''. В качестве примера можно привести
рисунок (\ref{fig:850_150}). Это обстоятельство позволяет увеличивать
скорость операций, увеличивая частоту Раби. Однако при превышении
частотой Раби резонансной частоты перехода, паразитная заселенность
резко и быстро возрастает с ростом частоты Раби, как это видно из
рисунка (\ref{fig:850_680}). Расчет заселенностоей уровней NV-центра
при больших частотах Раби ранее производился в работе
[\ref{lit:gigahertz}]. Экспериментальные и теоретические результаты,
полученные в работе [\ref{lit:gigahertz}] показаны на рисунке
(\ref{fig:gigahertz}). Можно заметить, что графики из данной работы и
работы [\ref{lit:gigahertz}] качественно соответствуют друг другу.

Чтобы учесть неидеальность резонансного излучателя, а также
дальнейшего увеличения скорости операции NOT, были произведены расчеты
для различных огибающих резонансного излучения. В работе
[\ref{lit:gigahertz}] были рассмотрены прямоугольные импульсы и
импульсы в форме гауссиана. Последние показали очень хорошие
результаты: операцию NOT удалось произвести менее, чем за
наносекунду. В данной работе была произведена попытка отыскать еще
более удачную форму огибающей микроволнового импульса. Здесь
показаны наиболее удачные формы: прямоугольные импульсы (\ref{fig:850_170}), импульсы в
форме гауссиана (\ref{fig:gauss}) и пилообразные импульсы
(\ref{fig:saw}). Качественное соответствие результатов, полученных при
расчетах в данной работе (\ref{fig:850_gigahertz}) и экспериментальных
данных (\ref{fig:gigahertz}), полученных в работе [\ref{lit:gigahertz}],
говорит о соответствии использованной модели реальному NV-центру. В результате
исследования различных форм было выявлено, что лучшей из исследованных
является гауссовидная форма амплитуды импульса. Это еще раз подтверждает
аналогию результатов исследований проведенных в данной работе и работе
[\ref{lit:gigahertz}].


Полученные времена проведения операции NOT дают основания полагать,
что NV-центры могут стать хорошим материалом для создания кубитов в
реальных квантовых компьютерах. Такие времена операций в сравнении с
временами когерентности даже при комнатных температурах дают основания
говорить о возможности производить сотни тысяч или даже миллионы
операций над кубитом до разрушения когерентного состояния.
