\newpage
\chapter*{Приложение}
\addcontentsline{toc}{chapter}{Приложение}
Для моделирования эволюции заселенностей NV-центра была использована
программа (\ref{code}). Расчет производился на основе
гамильтониана (\ref{eq:shamiltonian}). Форма, амплитуда и частота микроволнового
импульса подавались на вход программе, а на выходе получались данные
зависимости заселенностей уровней от времени. По этим данным были
получены представленные в работе графики.

В данном листинге представлен вариант программы, когда амплитуда
постоянного поля составляет 850 Гс. Здесь также учтены продольная и
поперечная компоненты тензора энергии расщепления в кристаллическом
поле электронных спиновых состояний.
\begin{lstlisting}[label=code, caption=NV-center model]
% @field_amplitude -- function(t) of amplitude of microwave field
% @omega -- frequency of microwave field
% @t_end -- time period during which the probabilities are calculated
%     in nanosecs
% return value -- matrix: vector t of time values, vector psi of
%     psi^2(probabilities)
function [t, psi] = test_func(field_amplitude, A, omega, t_end)
    t_end = t_end * 10^-9;
    B_z = 850;
    psi_init = [0 1 0];
    D = 2877 * 10^6;
    E = 7.7 * 10^6;
    % electron
    mu = 2.0028 * 5.788 * 10^-9;
    [t, psi] = nv_system(psi_init, D, E, mu, B_z, field_amplitude, 
                         A, omega, t_end);

    function [t, psi] = nv_system(psi_init, D, E, mu, B_z, B_0, 
                                  A, omega_field, t_end)
        h = 4.135 * 10^-15;
        t_init = 0;
        t_interval = [t_init, t_end];
        % Schrodinger equation    
        [t, psi] = ode45(@Right_part, t_interval, psi_init);
        [m,] = size(psi);
        % calculating probabilities
        for i = 1:m
            psi(i, 1) = abs(psi(i,1)^2);
            psi(i, 2) = abs(psi(i,2)^2);
            psi(i, 3) = abs(psi(i,3)^2);
        end;
    
        function dy = Right_part(t, y)
            k1 = h * D;
            k2 = h * E;
            Hamiltonian = [k1 + mu * B_z, 
                           mu * B_0(t, A) * cos(omega_field * t), 
                           2 * k2; 
                           mu * B_0(t, A) * cos(omega_field * t), 
                           0, 
                           mu * B_0(t, A) * cos(omega_field * t);
                           2 * k2, 
                           mu * B_0(t, A) * cos(omega_field * t), 
                           k1 - mu * B_z];
            dy = (Hamiltonian * y).*(-1i / h);
        end
    end
end
\end{lstlisting}
